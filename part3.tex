\section{Build a topic map of where there are gaps in my research area}

The single-cell 3D genome structure reconstruction area has recently grown considerably in the last four years. Also known as single-cell Hi-C (scHi-C), the technologies have enabled the analysis of 3D genome organization at an unprecedented resolution with a readout of chromatin interactions in individual cells. These new sequencing-based methods allow researchers to unveil the dynamics of chromatin conformation at a single-cell level and establish connections between structure and function obscured by population-level genomic mapping methods.

Survey papers that introduce and compare population-level genomic mapping methods, or 3D full genome reconstruction computation methods, are abundant in the field. Because there are many methods already discussed, analyzed, tested, and compared. But for single-cell Hi-C, there are still quite few of them because this sub area is still growing in popularity and interest. And it follows the advancements on the biological field of the actual biochemistry sequencing processes.

The fact that Hi-C data is noisy, coupled with other technical factors, makes it extremely difficult to determine the various unique 3D structures of cells used in Hi-C experiments. Due to the drawbacks involved in using multi-cell Hi-C data, studying single-cell Hi-C data has become increasingly relevant. In particular, it does not require designing an algorithm to satisfy the variability of each cell used in the Hi-C experiment. As expected, single-cell Hi-C datasets are sparser than multi-cell Hi-C datasets. Hence, conventional distance- and restraint-based methods are not suitable for 3D structure reconstruction based on these data.

In conclusion, the are of single-cell Hi-C (scHi-C) 3D genome structure reconstruction is still growing and need more knowledge spreading papers, such as survey papers, to bring more people to it and facilitate the developments of new and improved computational methods. That is why I believe it is a gap worth to explore and write survey papers about.