\section{Critically/creatively read survey papers}

I selected these 5 papers to critically/creatively read for this assignment. I added the information about number of citations and the year they were released:
\begin{enumerate}
    \item \cite{oluwadare2019overview} An overview of methods for reconstructing 3-D chromosome and genome structures from Hi-C data. \textbf{(cited by 103 since 2019)}
    \item \cite{mackay2020computational} Computational methods for predicting 3D genomic organization from high-resolution chromosome conformation capture data. \textbf{(cited by 21 since 2020)}
    \item \cite{li2020comparison} Comparison of computational methods for 3D genome analysis at single-cell Hi-C level. \textbf{(cited by 12 since 2020)}
    \item \cite{zhang2020advances} Advances in technologies for 3D genomics research. \textbf{(cited by 12 since 2020)}
    \item \cite{ulianov2022two} The two waves in single-cell 3D genomics. \textbf{(cited by 8 since 2022)}
\end{enumerate}

Below I provide summary and notes about the selected papers:

\begin{enumerate}
    \item This paper provides a comprehensive survey of contemporary methodologies employed for the reconstruction of three-dimensional chromosome and genome structures from Hi-C data. The authors systematically review various computational techniques and algorithms developed for inferring spatial chromatin organization, focusing on their underlying principles, strengths, and limitations. They discuss diverse approaches, including probabilistic modeling, dimensionality reduction, and graph-based representations, highlighting how each contributes to deciphering the complex spatial arrangements of chromosomes within the cell nucleus. Additionally, the paper underscores the significance of considering biological constraints and integrating multi-omics data to refine and validate reconstructed structures. This overview serves as a valuable resource for researchers in the field, offering insights into the evolving landscape of 3-D genome reconstruction techniques and their potential applications in advancing our understanding of genome organization and function.
    \item This paper offers a comprehensive review and evaluation of computational methods aimed at predicting the three-dimensional genomic organization from high-resolution chromosome conformation capture (Hi-C) data. The authors systematically assess various computational techniques employed in this domain, emphasizing their algorithms, data preprocessing steps, and applications. They provide critical insights into the strengths and limitations of these methods, shedding light on their performance in accurately deciphering the intricate spatial arrangements of the genome. Furthermore, the paper highlights the importance of considering biological context and incorporating multi-omics data to refine predictions. In sum, this review provides a valuable resource for researchers and practitioners in genomics, offering a clear overview of the state-of-the-art computational tools available for predicting 3D genomic organization from high-resolution Hi-C data.
    \item This paper presents a comprehensive comparative analysis of computational methods designed for 3D genome analysis at the single-cell Hi-C level. The authors systematically evaluate the performance of various tools, focusing on their ability to accurately capture chromatin interactions and structural features in single-cell data, while also considering factors like computational efficiency and ease of use. Through a rigorous benchmarking process, they highlight the strengths and weaknesses of each method and provide insights into their suitability for different research questions and datasets. This comparative study serves as a valuable resource for researchers working with single-cell Hi-C data, aiding them in selecting the most appropriate computational tools for their specific analytical needs and shedding light on the current state of the field in terms of 3D genome analysis at the single-cell level.
    \item This paper presents an overview of recent technological advancements in the field of 3D genomics research. It discusses various innovative techniques and tools that have emerged in recent years, enabling scientists to probe the three-dimensional organization of the genome with unprecedented precision. These advancements encompass improvements in chromatin conformation capture methods, such as Hi-C, as well as cutting-edge imaging and microscopy approaches, each offering unique insights into genome architecture. The paper emphasizes how these technological breakthroughs have facilitated the study of genome folding, spatial interactions, and their functional implications, ultimately contributing to a deeper understanding of gene regulation and genome function. Furthermore, it underscores the potential impact of these advancements on fields like epigenetics, development, and disease research, positioning 3D genomics as a rapidly evolving and transformative area of study within molecular biology.
    \item In this review, the authors summarize the present state of studies into 3D genome organization in individual cells, analyze the technical problems of single-cell studies, and outline perspectives of 3D genomics. This paper introduces the concept of "the two waves" in single-cell 3D genomics, delineating the evolution of this field into distinct phases. The first wave is characterized by the development of technologies like single-cell Hi-C and single-cell ATAC-seq, which enable the exploration of chromatin accessibility and interactions at the single-cell level. These methods have provided foundational insights into cellular heterogeneity and genome organization. The second wave builds upon these foundations and extends into advanced techniques, such as single-cell multi-omics and live-cell imaging, aiming to capture dynamic aspects of genome regulation and spatial organization in individual cells. The paper underscores the importance of recognizing these two waves and their interplay, highlighting the opportunities and challenges they present for understanding the intricacies of 3D genomics at the single-cell level, ultimately advancing our knowledge of genome function and regulation in diverse biological contexts.
\end{enumerate}