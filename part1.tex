\section{Define research area and search for papers in the area}

\subsection{Research area}
My research area in the Bioinformatics field is 3D genome structure reconstruction computational method development focused on single cells.

\subsection{Search for survey papers}
For this search, I used the terms \textbf{``single-cell 3D genome structure reconstruction algorithms survey OR review''}, filtered by date inside the last 7 years, to obtain a specific list of related survey/review papers in my area. I also included a few papers that my advisor suggested me to review. I selected the papers that I deemed more interesting and related to my area and added to a Google Scholar library label I created for this assignment. Then I exported the library in BiBTeX to add to my paper references. This is the list I selected:
\begin{itemize}
    \item \cite{oluwadare2019overview} An overview of methods for reconstructing 3-D chromosome and genome structures from Hi-C data.
    \item \cite{mackay2020computational} Computational methods for predicting 3D genomic organization from high-resolution chromosome conformation capture data.
    \item \cite{zhou20213d} The 3D genome structure of single cells.
    \item \cite{li2023techniques} Techniques for and challenges in reconstructing 3D genome structures from 2D chromosome conformation capture data.
    \item \cite{li2020comparison} Comparison of computational methods for 3D genome analysis at single-cell Hi-C level.
    \item \cite{kos2021perspectives} Perspectives for the reconstruction of 3D chromatin conformation using single cell Hi-C data.
    \item \cite{fang2022mapping} Mapping nucleosome and chromatin architectures: A survey of computational methods.
    \item \cite{moon2018manifold} Manifold learning-based methods for analyzing single-cell RNA-sequencing data.
    \item \cite{galitsyna2021single} Single-cell Hi-C data analysis: safety in numbers.
    \item \cite{zhang2020advances} Advances in technologies for 3D genomics research.
    \item \cite{ulianov2022two} The two waves in single-cell 3D genomics.
    \item \cite{wen2022recent} Recent advances in single-cell sequencing technologies.
\end{itemize}

\subsection{Git availability}

This journal is available in a public Git repository I created in my GitHub account via the integration with Overleaf:
\begin{itemize}
    \item \url{https://github.com/cetorres/CS-6000---Assignment---Journal-3}
\end{itemize}